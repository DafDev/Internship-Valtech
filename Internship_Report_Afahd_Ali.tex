\documentclass[a4paper,12pt]{article}

%------------------------------------------------------------
%					Languages packages
%------------------------------------------------------------
\usepackage[utf8]{inputenc} %utilisation de caract�res lati et codage UTF-8
\usepackage[T1]{fontenc}
\usepackage{palatino}     % Police utilis�e
\usepackage[frenchb]{babel} %Dico fran�ais

%-------------------------------------------------------------
%					Picture packages
%-------------------------------------------------------------

\usepackage{graphicx} %inclusion d'image avec l'environnement figure
\usepackage[lofdepth,lotdepth]{subfig}
\usepackage{here}
\usepackage{framed}
\usepackage{color}
\definecolor{lightgray}{gray}{0.5}
\definecolor{myblue}{rgb}{0.2,0,0.8} % d�inition perso de couleur dans la norme RGB
\definecolor{shade}{gray}{0}
\definecolor{mygreen}{rgb}{0,0.6,0}
\definecolor{purple}{rgb}{0.77,0,0.55}



%-------------------------------------------------------------
%				Mathematics packages
%-------------------------------------------------------------

\usepackage{amsmath}
\usepackage{cases} %permet de faire des syst�mes d'�quations
\usepackage{amssymb}
\usepackage{mathrsfs}
\usepackage{moreverb}


%------------------------------------------------------------------
%				 Coding packages
%------------------------------------------------------------------
\usepackage{textcomp}
\usepackage{listings}
\lstset{ 
language=Matlab,        % choix du langage
tabsize=2,											%taile des tabulations
basicstyle=\ttfamily,       % taille de la police du code
keywordstyle=\ttfamily\color{myblue} ,    %couleur des mots-cl?s
columns=flexible,
upquote=true,
numbers=left,                   % placer le num?ro de chaque ligne ? gauche (left) 
numberstyle=\normalsize,        % taille de la police des num?ros
numbersep=7pt,                  % distance entre le code et sa num?rotation
backgroundcolor=\color{white},  % couleur du fond Possibilit? d'utilisation du package color 
frame=single,										%ajout d'une bordure autour du code						
commentstyle=\color{mygreen},    % comment style
numberstyle=\tiny\color{shade}, % the style that is used for the line-numbers
rulecolor=\color{black},         % if not set, the frame-color may be changed on line-breaks within not-black text (e.g. comments (green here))
stringstyle=\color{purple},     % string literal style
title=\lstname                   % show the filename of files included with \lstinputlisting; also try caption instead of title
}
%On peut aussi utiliser la commande :
%\lstinputlisting[language=Python]{source_filename.py}
%pour inclure directement un fichier source dans le pdf

\usepackage{algorithm2e} %�crire du pseudo-code
\usepackage{lipsum}
\usepackage{hyperref} %cr�er des liens PDF 
\hypersetup{
    colorlinks = true,
    linkcolor=black,
    }
%----------------------------------------------------------
%					Page setting packages
%----------------------------------------------------------

\usepackage{titlesec}
\usepackage{bookmark}
%\titleformat{\chapter}[hang]{\bfseries}{\large\thechapter}{10pt}{\LARGE}
\titleformat{\chapter}[hang]{\bf\Large}{\thechapter}{2pc}{}
\titleformat{\section}[hang]{\bf\large}{\thesection}{2pc}{}
\usepackage[top=1.5cm, bottom=3.5cm,right=2cm,left=2cm]{geometry}
% \geometry
\usepackage{setspace} %pour regler l'interligne
\usepackage{textcomp}
%\setlength{\parindent}{1ex}%set indent
\setlength{\parskip}{2ex} %Space between paragraphs
\usepackage{tabularx} %enables complicated arrays
\usepackage{fancyhdr} %set pages styles with a lot more options
\pagestyle{fancy}
%\usepackage[square]{natbib}
%renomming \listoffgures and \listoftables  names to something which sound better in French
%\addto\captionsfrench{%
%  \renewcommand{\listfigurename}{Listes des figures}%
%  \renewcommand{\listtablename}{Liste des tableaux}%
%  \renewcommand{\contentsname}{Sommaire}%
%}
\renewcommand{\contentsname}{Sommaire}
%%%%%%%%%%%%%%%%%%%%%%%%%%%%%%%%%%%%%%%%%%%%%%%%%%%%%%%%%%%%%%%%%%%%%%%%%%%%%%%%%%%%%%%%%%%%%%%%%%%%%%%%%%%%%%%%%%%%%%%%%%%%%%%%%%%%%%%%%%%%%%%%%%%%%%%%%%%%%%%%%%
%
%															Begin Document
%
%%%%%%%%%%%%%%%%%%%%%%%%%%%%%%%%%%%%%%%%%%%%%%%%%%%%%%%%%%%%%%%%%%%%%%%%%%%%%%%%%%%%%%%%%%%%%%%%%%%%%%%%%%%%%%%%%%%%%%%%%%%%%%%%%%%%%%%%%%%%%%%%%%%%%%%%%%%%%%%%%%

\begin{document}
%Page de garde de ouf^^

\begin{titlepage}
    \newcommand{\HRule}{\rule{\linewidth}{0.7mm}} % Defines a new command for the horizontal lines, change thickness here
    
    \center % Center everything on the page
    
    %----------------------------------------------------------------------------------------
    %	HEADING SECTIONS
    %----------------------------------------------------------------------------------------
    
    \textsc{\LARGE Efrei Paris \- Valtech\_}\\[1.5cm] % Name of your university/college
    %\textsc{}\\[0.5cm] % Major heading such as course name
    %\textsc{\large Project report}\\[0.5cm] % Minor heading such as course title
    
    %----------------------------------------------------------------------------------------
    %	TITLE SECTION
    %----------------------------------------------------------------------------------------
    
    \HRule \\[0.4cm]
    { \huge \bfseries Rapport de Stage }\\[0.4cm] % Title of your document
    \HRule \\[1.5cm]
    
    %----------------------------------------------------------------------------------------
    %	AUTHOR SECTION
    %----------------------------------------------------------------------------------------
    
    \begin{minipage}{0.4\textwidth}
    \begin{flushleft} \large
    \emph{Référent \'Ecole :}\\
    Leon \textsc{Evain} % Your name
    \end{flushleft}
    \end{minipage}
    ~
    \begin{minipage}{0.4\textwidth}
    \begin{flushright} \large
    \emph{Maître de stage :}\\
    Pedro \textsc{Roche}\\ % Supervisor's Name
    Global EMEA Head Delivery
    \end{flushright}
    \end{minipage}\\[4cm]
    
    % If you don't want a supervisor, uncomment the two lines below and remove the section above
    \Large \emph{Etudiant :}\\
    Afahd \textsc{Ali}\\[2cm] % Your name
    
    
    
    %----------------------------------------------------------------------------------------
    %	LOGO SECTION
    %----------------------------------------------------------------------------------------
    
    \includegraphics[width=0.5\textwidth]{logo-efrei-paris.jpg}\\[2.5cm] % Include a department/university logo - this will require the graphicx package
    
    %----------------------------------------------------------------------------------------
    %	DATE SECTION
    %----------------------------------------------------------------------------------------
    
    %{\large \today} % Date, change the \today to a set date if you want to be precise 
    %----------------------------------------------------------------------------------------
    
    \vfill % Fill the rest of the page with whitespace
    
\end{titlepage}
%--------------------------------------
% ce que tu veux mettre dans les  
% en-têtes et pieds de pages
%--------------------------------------
\renewcommand{\headheight}{5pc} % Espace entre l'en-t�te et le haut de la feuille
\renewcommand{\headrulewidth}{1ex} % Trait en haut
\renewcommand{\headsep}{2pc}
\renewcommand{\footrulewidth}{1ex} % Trait en bas

% en-t�te
\fancyhead[L]{\includegraphics[width=0.1\textwidth]{logo-efrei-paris.jpg}} %coin gauche 
\fancyhead[C]{} %centre
\fancyhead[R]{Page \thepage } %� droite
%pied de pages
\fancyfoot[L]{Afahd \textsc{Ali} - SE}
\fancyfoot[C]{}
\fancyfoot[R]{}


\tableofcontents 
\newpage
\section{Introduction}
Ce rapport concerne mon stage éffectué à Valtech entre le 5 mars et 7 septembre 2018.
Le sujet de ce rapport était une formation à Sitecore puis une intégration à une équipe client.

\section{Talent Program}
\label{sec:talent-program}
Le \emph{Talent Program} que m'a proposé Valtech est un
programme de formation sur les différentes plateformes utilisées aux seins
des porjets clients de l'agence.

J'ai suivi le \emph{Talent Program} dédié à Sitecore
%%%%%%%%%%%%%%%%%%%%%%%%%%%%%%%%%%%%%%%%%%%%%%%%%%%%%%%%%%%%%%%%%%%%%%%%%%%%%%%%
%								New Chapter																			 
%
%%%%%%%%%%%%%%%%%%%%%%%%%%%%%%%%%%%%%%%%%%%%%%%%%%%%%%%%%%%%%%%%%%%%%%%%%%%%%%%%


%%%%%%%%%%%%%%%%%%%%%%%%%%%%%%%%%%%%%%%%%%%%%%%%%%%%%%%%%%%%%%%%%%% new section

%--------------------------------------------------------new subsection


%%%%%%%%%%%%%%%%%%%%%%%%%%%%%%%%%%%%%%%%%%%%%%%%%%%%%%%%%%%%%%%%%%% new section

\end{document}